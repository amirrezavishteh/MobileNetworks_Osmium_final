\documentclass[]{report}

\usepackage[affil-it]{authblk}
\usepackage{hyperref}
\usepackage{graphicx}
\hypersetup{
	colorlinks=true
}
\usepackage{xepersian}

\settextfont[Scale=1.6]{Microsoft Uighur}




\author{امیر رضا ویشته 99521388}
\affil{محمدحسین رحیمی 99521298\\}

\title{کارهایی که برای انجام پروژه Osmium انجام دادیم}

\affil{درس شبکه های تلفن همراه}
\affil{دانشگاه علم و صنعت ایران}





\begin{document}

\maketitle
	\tableofcontents
	
\newpage

\section{راهکار اول}

در راهکار اول ما با استفاده از توان سیگنال‌های دریافتی (RSSI) و اطلاعات مکانی (GPS) برای تخمین مکان سلول‌های شبکه تلفن همراه گام برداشتیم. در این راهکار، برنامه اندروید سیگنال‌های دریافتی را جمع‌آوری می‌کند و با استفاده از الگوریتم‌های مختلف مکان‌یابی، مکان سلول‌های شبکه را تخمین می‌زند.
مراحل:\\
1.	دریافت permission ها: برنامه مجوزهای لازم برای دسترسی به سیگنال‌های تلفن همراه و مکان دستگاه را درخواست می‌کند.\\
2.	جمع‌آوری اطلاعات سلولی: برنامه اطلاعات سلولی از جمله قدرت سیگنال و اطلاعات شناسایی سلول را جمع‌آوری می‌کند.\\
- دریافت اطلاعات سیگنال از سلول‌های اطراف از جمله قدرت سیگنال (dBm).
- جمع‌آوری شناسه‌های سلول (Cell IDs) و سایر اطلاعات مرتبط مانند LAC (Location Area Code) وCell Id

- GPS
- دریافت اطلاعات مکانی شامل طول جغرافیایی (Longitude) و عرض جغرافیایی (Latitude).\\


3.	محاسبه قدرت سیگنال: سیگنال‌های دریافتی از سلول‌های مختلف تجزیه و تحلیل می‌شوند تا قدرت سیگنال هر سلول مشخص شود.
- قدرت سیگنال دریافتی با فاصله از منبع سیگنال رابطه معکوس دارد.
- از مدل‌های انتشار سیگنال (مثل مدل log-distance) استفاده می‌شود تا فاصله تقریبی از هر سلول را بر اساس RSSI تخمین بزنیم.\\
4.	ارسال پیامک و دریافت پیامک: برنامه پیامک ارسال و دریافت می‌کند.

\begin{center}
	\includegraphics[width=12cm]{img1}
	\includegraphics[width=12cm]{img2}
	\includegraphics[width=12cm]{img3}
	\includegraphics[width=12cm]{img4}
	
\end{center}

\section{راهکار دوم}
راهکار دوم ما شامل استفاده از الگوریتم پهلوبندی دایره‌ای \lr{(Circular Lateration)} برای تخمین موقعیت هدف است. این روش با استفاده از موقعیت‌ها و فاصله‌های اندازه‌گیری شده بین نقاط مختلف، مکان دقیق هدف را محاسبه می‌کند.
مراحل:\\
1.	ایجاد ماتریس‌ها: ماتریس‌های لازم برای انجام محاسبات پهلوبندی دایره‌ای را ایجاد می‌کنیم\\

- از ماتریس‌های زیر برای حل معادلات استفاده می‌شود:
\begin{center}
	\includegraphics{img5}

	
\end{center}

2.	محاسبه موقعیت: با استفاده از ماتریس‌ها و حل معادلات خطی، موقعیت دقیق هدف محاسبه می‌کنیم.
- با استفاده از روش‌های مختلف حل معادلات خطی (مانند روش حداقل مربعات)، موقعیت تخمین زده شده را پیدا می‌کنیم:
\begin{center}
	\includegraphics[width=12cm]{img5}
	\includegraphics[width=12cm]{img6}
	\includegraphics[width=12cm]{img7}
	
	
\end{center}
تابعlateration  به منظور تخمین موقعیت یک نقطه هدف در یک سیستم مختصات با استفاده از موقعیت‌ها و فاصله‌های اندازه‌گیری شده از چندین گره راهنما طراحی شده است.

ورودی‌ها:
1. positions: 
- یک آرایه دو بعدی از نوع double که مختصات گره‌های راهنما را نشان می‌دهد.
- هر عنصر در این آرایه یک آرایه دو عنصری است که مختصات x و y یک گره راهنما را مشخص می‌کند.

2. distances: 
- یک آرایه از نوع double که فاصله‌های گره‌های راهنما تا گره هدف را نشان می‌دهد.
- هر عنصر در این آرایه یک فاصله است که مربوط به گره راهنمای متناظر در آرایه `positions` است.
- مثلاً distances[i] فاصله‌ی گره راهنمای i تا گره هدف است.

خروجی:
- یک آرایه از نوع  double که شامل مختصات تخمینی گره هدف است.
- این آرایه دو عنصر دارد که عنصر اول مختصات `x` و عنصر دوم مختصات `y` گره هدف را نشان می‌دهد.

عملکرد:


1. ساخت ماتریس‌ها:
- یک ماتریس A و یک بردار B ساخته می‌شوند که برای حل سیستم معادلات خطی لازم هستند.(قبلا به فرم انها اشاره شد.)
- ماتریس A با استفاده از تفاوت مختصات گره‌ها و بردار B با استفاده از فاصله‌ها و مختصات محاسبه می‌شود.

2. حل سیستم معادلات:
- با استفاده از ماتریس A و بردار B، سیستم معادلات خطی Ax = B حل می‌شود تا مختصات گره هدف تخمین زده شود.



\section{باقی بخش ها}
توابعی که در بالا توضیح داده شدند در پوشه utils قرار دارند.
کد اصلی برنامه در Main Activity توسعه داده شده است.
ما ابتدا دسترسی های لازم را از کاربر میگیریم.
سپس در ادامه برای دسترسی به اطلاعات سلول های اطراف از TelephonyManager استفاده کردیم و با استفاده از آن توان سیگنال و دیگر اطلاعات را بدست آوردیم و با استفاده از توابع توسعه داده شده و با استفاده از پهلو بندی دایره ای و دیگر محاسبات اطلاعات مورد نیاز را بدست آوردیم. این کار را به طور مداوم انجام میدهیم تا اطلاعات را به صورت Time Real داشته باشیم.
برنامه ما دو تب دارد که در هر کدام اطلاعات مربوطه نمایش داده میشوند.
\begin{center}
	\includegraphics[width=12cm]{img9}
	\includegraphics[width=12cm]{img10}
	
	\end{center}
	در صفحه سیگنال های دریافتی، با توجه به اطلاعات دریافت شده از سلول ها، آیدی هر سلول و توان آن را نمایش میدهیم.
	در هنگام نوشتن این گزارش، گوشی من به باگی در پرمیشن ها خورد (مطمئن نیستم ولی اینطور فکر میکنم چون لاگ های دریافتی خالی بودند)و دسترسی ها به درستی داده نمیشد و نتوانستم عکسی از اطلاعاتی که نمایش میدهیم بگذارم اما در اولین فرصت براتون ارسال میکنم.\\
	در صفحه سلول ها، ما موقعیت سلول هایی که محاسبات  لازم را برای آن ها انجام دادیم نمایش میدهیم.
	

\end{document}
